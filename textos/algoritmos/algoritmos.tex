%% LyX 2.2.3 created this file.  For more info, see http://www.lyx.org/.
%% Do not edit unless you really know what you are doing.
\documentclass[english]{article}
\usepackage[T1]{fontenc}
\usepackage[latin9]{inputenc}
\usepackage{algorithm2e}

\makeatletter
%%%%%%%%%%%%%%%%%%%%%%%%%%%%%% User specified LaTeX commands.
\usepackage{algorithmic}

\makeatother

\usepackage{babel}
\begin{document}

\title{Algoritmos}

\author{Wandeson Ricardo\\
}

\maketitle
\newpage{}

\tableofcontents{}

\newpage{}

Anota��es do livro Algoritmos de Cormmen.

\newpage{}

\section{Introdu��o}

\subsection{Algoritmos}

Informalmente algoritmo � qualquer procedimento computacional bem
definido que toma algum valor ou conjunto de valores como entrada
e produz algum valor ou conjunto de valores como sa�da. Portanto um
algoritmo � uma sequ�ncia de passos computacionais que transformam
a entrada em sa�da.

\pagebreak{}

\subsection{Conceitos B�sicos}

\subsubsection{Ordena��o por inser��o}

Nos primeiro algoritmo, o de ordena��o por inser��o

\begin{algorithm}

\begin{algorithm}[H]  
\KwData{this text}  
\KwResult{how to write algorithm with \LaTeX2e }  
initialization\;  
\While{not at end of this document}{   
read current\;   
\eIf{understand}{    
go to next section\;    
current section becomes this one\;    
}
{    go back to the beginning of current 
section\;  

}  
} 

\caption{How to write algorithms} 
\end{algorithm}


\end{algorithm}

\pagebreak{}

\section{Crescimento de Fun��es e Analise}

\subsection{Recorr�ncias}

\subsection{Analise probabilistica e algoritmos aleat�rios}

\pagebreak{}

\section{Estrutura de Dados}
\end{document}
